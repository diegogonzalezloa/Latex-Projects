% ##############
% Codificación para el idioma español
% ##############
%\usepackage[utf8]{inputenc} % input
\usepackage[T1]{fontenc} % output
%\usepackage[spanish,english]{babel} % idioma español para tablas, figuras, secciones, etc.

% ##############
% Dimensiones del documento
% ##############
\usepackage{geometry}
\geometry{
    letterpaper,
    margin = .5in,
    marginparsep = 0.1in,
    marginparwidth = 0.3in,
    headheight = 0.3in,
    headsep = 0.1in,
    footskip = 0.35in,
}
\setlength{\columnsep}{.2in}

% Pages Styles
\pagestyle{empty}

% ##############
% Formato de parrafos
% ##############
\setlength{\parskip}{.5\baselineskip}
\setlength{\parindent}{0pt}

% ##############
% Modificar el estilo de texto del documento
% ##############

% Estilo de Fonts
\usepackage{fontspec}

\setromanfont{Author}[
    Path=./FontFiles/Author/,
    Extension = .otf,
    UprightFont=*-Regular,
    BoldFont=*-Bold,
    ItalicFont=*-Italic,
    BoldItalicFont=*-BoldItalic
    ]

\newfontfamily{\fnamefont}{DancingScript}[
    Path=./FontFiles/DancingScript/,
    Extension = .otf,
    UprightFont=*-Regular,
    BoldFont=*-Bold
    ]

\newfontfamily{\snamefont}{CutiveMono}[
    Path=./FontFiles/CutiveMono/,
    Extension = .ttf,
    UprightFont=*-Regular
    ]
    
\newfontfamily{\sectionfont}{PublicSans}[
    Path=./FontFiles/PublicSans/,
    Extension = .otf,
    UprightFont=*-Regular,
    BoldFont=*-Bold
    ]

% Escala del Font size
\usepackage{scalefnt}

% ##############
% Xcolor
% ##############
\usepackage{xcolor}
\definecolor{fillcolor}{HTML}{ccc5b9}
\definecolor{fillcolor2}{HTML}{eb5e28}
\definecolor{linkcolor}{HTML}{eb5e28}

% ##############
% Calc
% ##############
\usepackage{calc}

% ##############
% Tabla
% ##############
\usepackage{array}
\newcolumntype{L}[1]{>{\raggedright\arraybackslash}p{\dimexpr#1\linewidth-2\tabcolsep}}
\newcolumntype{R}[1]{>{\raggedleft\arraybackslash}p{\dimexpr#1\linewidth-2\tabcolsep}}
\newcolumntype{J}[1]{>{\justifying\arraybackslash}p{\dimexpr#1\linewidth-2\tabcolsep}}
\newcolumntype{C}[1]{>{\Centering\arraybackslash}p{\dimexpr#1\linewidth-2\tabcolsep}}

% ##############
% Imagenes y Figuras
% ##############
\usepackage{graphicx} % para insertar imagenes en Latex
\graphicspath{Images/} % le dice a latex donde están guardadas las figuras

% ##############
% Tikz
% ##############
\usepackage{tikz}

% ##############
% SVG
% ##############
\usepackage[inkscapeformat=eps]{svg}

% ##############
% Adjust Box
% ##############
\usepackage{adjustbox}

% ##############
% Hipervinculos
% ##############
\usepackage[hidelinks]{hyperref}
% \hypersetup{
%     pdftitle={DiegoGonzalez_CV},
%     pdfauthor={Diego Gonzalez}
% }
% Url personalizado
\newcommand{\myurl}[1]{\adjustbox{cframe=linkcolor {1.5pt} {1pt} {-1.5pt-1pt}}{\url{#1}}}
\newcommand{\myhref}[2]{\adjustbox{cframe=linkcolor {1.5pt} {2pt} {-1.5pt-2pt}}{\href{#1}{#2}}}
\newcommand{\extlink}[2]{\adjustbox{stack=cc, cframe=linkcolor {1.5pt} {1pt} {-1.5pt-1pt}}{\href{#1}{\includesvg[width=2ex]{SVG/external-link.svg}}}}

% ##############
% My Items
% ##############
\newcommand{\myitem}{\adjustbox{stack=cc}{\includesvg[width=2ex]{SVG/check-circle.svg}}~~}

% ##############
% Overall Skill
% ##############
\newcommand{\overallskill}[1]{
    \adjustbox{stack=cc}{
    \begin{tikzpicture} [remember picture, rounded corners]
        \draw [fill=fillcolor,draw=none] (0,0) rectangle (\linewidth+2\tabcolsep,.8em);
        \draw [fill=fillcolor2,draw=none] (0,0) rectangle (#1\linewidth+2\tabcolsep,.8em);
    \end{tikzpicture}}
}

% ##############
% My Section
% ##############
\newcommand{\mysection}[1]{
    \vspace{1.5\parskip}
    \begin{tikzpicture} [remember picture]
        \node [
            rectangle, 
            rounded corners=8pt, 
            text width=\linewidth-8pt, 
            inner sep=4pt,
            align=center,
            fill=fillcolor
            ] {\sectionfont\small \textbf{#1}};
    \end{tikzpicture}\par
}

\newcommand{\mysectiontwo}[1]{
    \vspace{1.5\parskip}
    \begin{tikzpicture} [remember picture]
        \node [
            rectangle, 
            rounded corners=8pt, 
            text width=\linewidth-8pt, 
            inner sep=4pt,
            align=center,
            fill=white
            ] {\sectionfont\small \textbf{#1}};
    \end{tikzpicture}\par
}

% ##############
% Background
% ##############
\newcommand{\backgorund}{
    \begin{tikzpicture} [remember picture, overlay]
        \node at (0pt,13pt) [
            % below right,
            anchor = north west,
            fill = fillcolor,
            rounded corners = .25\columnwidth,
            minimum width=\columnwidth,
            minimum height=\textheight
        ] {};
    \end{tikzpicture}
}

% ##############
% Paquetes de Visualización para crear el documento
% ##############
% \usepackage{layout}
% \usepackage{showframe}
% \usepackage{lipsum}