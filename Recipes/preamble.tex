% ##############
% Paquetes de Visualización para crear el documento
% ##############
% \usepackage{layout}
% \usepackage{showframe}
\usepackage{lipsum}

% ##############
% Metadata
% ##############
\logo{%
    \includegraphics[scale=0.45]{logo_usb}
}
\titlehead{Proyecto de (Materia)}
\newcommand{\titulo}{Titulo del Documento IEEE}
\newcommand{\autor}{%
    Autor Uno\authorrefmark{1},
    Autor Dos\authorrefmark{2}, 
    Autor tres\authorrefmark{3} y
    Autor Cuatro\authorrefmark{4}
}
\lname{Apellido/os}
\newcommand{\facultad}{Departamento o Facultad}
\newcommand{\universidad}{Universidad o Institución}
\newcommand{\ciudad}{Ciudad}
\newcommand{\pais}{País}
\newcommand{\dirEmail}{%
    \authorrefmark{1}
        \href{malito:autor.uno@añadir.red}{autor.uno@añadir.red},
    \authorrefmark{2}
        \href{malito:autor.dos@añadir.red}{autor.dos@añadir.red},
    \authorrefmark{3}
        \href{malito:autor.tres@añadir.red}{autor.tres@añadir.red},
    \authorrefmark{4}
        \href{malito:autor.cuatro@añadir.red}{autor.cuatro@añadir.red}
}

% ##############
% Codificación para el idioma español
% ##############
% \usepackage[utf8]{inputenc}
\usepackage[T1]{fontenc} % output

% ##############
% Dimensiones del documento
% ##############
\usepackage{geometry}
\geometry{%
    letterpaper,
    margin = .5in,
    marginparsep = 0.1in,
    marginparwidth = .3in,
    headsep = .1in,
    headheight = .3in,
    footskip = .4in
}
\pagestyle{empty}

% ##############
% Formato de parrafos
% ##############
\setlength{\parskip}{.5\baselineskip}
\setlength{\parindent}{0pt}
\usepackage{setspace} % espaciado de los parrafos
\onehalfspacing % 1.5 de espaciado de linea

% ##############
% Modificar el estilo de texto del documento
% ##############

% Estilo de Fonts
\RequirePackage{fontspec}

\setromanfont{PaperNotes}[
    Path=./FontFiles/PaperNotes/,
    Extension = .ttf,
    UprightFont=*-Regular,
    BoldFont=*-Bold,
    ItalicFont=*-Sketch
    ]

\newfontfamily{\titlefont}{ChampagneLimousines}[
    Path=./FontFiles/ChampagneLimousines/,
    Extension = .ttf,
    UprightFont=*-Regular,
    BoldFont=*-Bold,
    ItalicFont=*-Italic,
    BoldItalicFont=*-BoldItalic
    ]

% Small caps
% Small Caps create
\usepackage{graphicx}

\makeatletter
\newlength\fake@f
\newlength\fake@c
\def\fakesc#1{%
  \begingroup%
  \xdef\fake@name{\csname\curr@fontshape/\f@size\endcsname}%
  \fontsize{\fontdimen8\fake@name}{\baselineskip}\selectfont%
  \uppercase{#1}%
  \endgroup%
}
\makeatother
\newcommand\fauxsc[1]{\fauxschelper#1 \relax\relax}
\def\fauxschelper#1 #2\relax{%
  \fauxschelphelp#1\relax\relax%
  \if\relax#2\relax\else\ \fauxschelper#2\relax\fi%
}
\def\Hscale{.8}\def\Vscale{.75}\def\Cscale{1.00}
\def\fauxschelphelp#1#2\relax{%
  \ifnum`#1=\lccode`#1\relax\scalebox{\Hscale}[\Vscale]{\char\uccode`#1}\else%
    \scalebox{\Cscale}[1]{#1}\fi%
  \ifx\relax#2\relax\else\fauxschelphelp#2\relax\fi}

% ##############
% Imagenes y Figuras
% ##############
\usepackage{graphicx} % para insertar imagenes en Latex
\graphicspath{ {./Images/} } % le dice a latex donde están guardadas las figuras
\usepackage{wrapfig}

% ##############
% Tablas
% ##############
\usepackage{array} % para poner las medidas del ancho del tabular
\usepackage{nicematrix} % paquete para rellenar las celdas con un color de background

% ##############
% Tikz
% ##############
\usepackage{tikz}

% ##############
% Calc
% ##############
\usepackage{calc}

% ##############
% Xcolor
% ##############
\usepackage{xcolor}

% ##############
% ifthen
% ##############
\usepackage{ifthen}

% ##############
% Comandos de Secciones
% ##############

% section
\newsavebox{\secimagbox}
\newcommand{\mysection}[6]{%
    \savebox{\secimagbox}{\includegraphics[width=#3]{#2}}
    \begin{wrapfigure}[#5]{#4}{\wd\secimagbox}
        \begin{tikzpicture} [remember picture]
            \node [
                fill,
                rounded corners = 10pt,
                inner sep=0pt,
                line width=0pt,
                minimum width=\wd\secimagbox,
                minimum height=\ht\secimagbox,
                path picture= {
                    \node at (path picture bounding box.center) {\usebox{\secimagbox}};
                }
            ] {};
        \end{tikzpicture}
    \end{wrapfigure}

    \vspace{-\parskip}
    \ifthenelse{\equal{#4}{l}} {%
        \textbf{\large\fauxsc{#1}}
        
        \begin{table}[htb!]
            \hspace{\wd\secimagbox+6pt}
            #6
        \end{table}
    }{%
        \raggedleft
        \textbf{\large\fauxsc{#1}}
        
        \begin{table}[htb!]
            \raggedleft
            #6
            \hspace{\wd\secimagbox+6pt}
        \end{table}
    }
    \raggedright
    \vspace{-.5\baselineskip}
}


% \newcommand{\foo}[1]{%
%   \ifthenelse{\equal{#1}{\string german}}
%     {TRUE}
%     {FALSE}%
% }