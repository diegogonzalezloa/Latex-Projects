\mysection{Arroz Simple} % Titulo
{rice.jpg}{.3\textwidth}{l}{6} % Argumentos de imagen
{ % Ingredientes
    \begin{tabular}[t]{@{}llll@{}}
        principales: & arroz & agua & sal \\
        opcionales: & tomate \\
    \end{tabular}
}

Las dos cosas mas importantes a la hora de preparar arroz son el tiempo y las proporciones según el tipo de arroz. Si cocinas el arroz por poco tiempo podría quedar duro o si por el contrario lo cocinas por demasiado tiempo puede que quede gomoso y blando. Con respecto a las proporciones, si el arroz tiene demasiada agua resultará húmedo o si tiene poca se quemará. La siguiente tabla puede guiarte sobre cómo deberías cocinar arroz según su tipo:

\begin{table}[htb!]
    \centering
    \begin{NiceTabular}{rccc}
    & \textbf{Cantidad de arroz} & \textbf{Cantidad de agua} & \textbf{Tiempo para hervir a fuego lento} \\[5pt]
    Arroz blanco de grano largo & 1 taza & 2 tazas & 18 min \\
    Arroz blanco de grano corto & 1 taza & 1 taza y 1/4 & 15 min \\
    Arroz integral & 1 taza & 1 taza y 3/4 & 45 min \\
    \CodeAfter
        \tikz
            \draw [dashed, opacity=.75, line width=1.25pt]
                (1-|2) -- (last-|2)
                (1-|3) -- (last-|3)
                (1-|4) -- (last-|4);
    \end{NiceTabular}
\end{table} \par

Antes de cocinar el arroz te aconsejo enjuagar el arroz para eliminar el exceso de almidón, pero es un paso opcional. Para cocinar el arroz, agregarlo en una olla junto con su respectiva proporción de agua. Agregar también una pizca de sal para darle sabor al arroz, la cantidad ya es a tu gusto. Opcionalmente puedes agregar tomate licuado para darle un toca mas de sabor. Déjalo hervir a fuego alto y una vez que el agua comience a hervir, revuelve, tapa la olla y baja el fuego para que hierva el arroz a fuego lento. Por ultimo cocinar el arroz a fuego lento durante el tiempo promedio que muestra la tabla anterior.