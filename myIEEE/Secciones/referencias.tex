\section*{Referencias}

Numere las citas consecutivamente entre corchetes \cite{young1989}. La puntuación de la frase sigue al corchete \cite{clerk1892}. Haga referencia simplemente al número de referencia, como en \cite{jacobs1963}- no utilice ``Ref. \cite{jacobs1963}'' o ``referencia \cite{jacobs1963}'' excepto al principio de una frase: ``La referencia \cite{jacobs1963} fue la primera...''. \par

Numere las notas a pie de página por separado en superíndices. Coloque la nota a pie de página propiamente dicha al final de la columna en la que se ha citado. No ponga notas a pie de página en el resumen ni en la lista de referencias. Utilice letras para las notas a pie de tabla. \par

A menos que haya seis autores o más, indique todos los nombres de los autores; no utilice ``et al.''. Los trabajos que no se hayan publicado aunque se hayan presentado para su publicación, deben citarse como ``no publicado'' [4]. Los trabajos aceptados para publicación se citarán como ``en prensa'' [5]. Escriba sólo en mayúsculas la primera palabra del título, excepto los nombres propios y los símbolos de elementos. \par

Para los artículos publicados en revistas de traducción, indique primero la cita en inglés, seguida de la cita original en el idioma extranjero [6].