% ##############
% Codificación para el idioma español
% ##############
%\usepackage[utf8]{inputenc} % input
\usepackage[T1]{fontenc} % output
%\usepackage[spanish,english]{babel} % idioma español para tablas, figuras, secciones, etc.
\usepackage[style=spanish]{csquotes} % Cuando se usa el paquete 'babel' o 'polyglossia' con 'biblatex', es recomendado cargar 'csquotes'  para asegurar que la tipografía de las comillas este correcta de acuerdo al idioma a utilizar.

% ##############
% Calc
% ##############
\usepackage{calc}

% ##############
% Dimensiones del documento
% ##############
\usepackage{geometry}
\geometry{
    letterpaper,
    margin = .7in,
    headheight = .3in,
    headsep = .2in,
    footskip = 0.2in+12pt, % 12pt son del tamaño de texto
    marginparwidth = .4in,
    marginparsep = .15in,
}
% Pages Styles
\pagestyle{empty}

% ##############
% Formato de parrafos
% ##############
\setlength{\parskip}{.5\baselineskip}
\setlength{\parindent}{0pt}

% ##############
% Xcolor
% ##############
\usepackage{xcolor}
\definecolor{fontcolor}{HTML}{353535}
\definecolor{fillcolor}{HTML}{3c6e71}
\definecolor{linkcolor}{HTML}{d9d9d9}

% ##############
% Modificar el estilo de texto del documento
% ##############

% Estilo de Fonts
\usepackage{fontspec}

\setmainfont{Author}[
    Color = fontcolor,
    Path=./FontFiles/Author/,
    Extension = .otf,
    Ligatures = TeX,
    UprightFont=*-Regular,
    BoldFont=*-Bold,
    ItalicFont=*-Italic,
    BoldItalicFont=*-BoldItalic
    ]

\newfontfamily{\titlefont}{Poppins}[
    Color = fontcolor,
    Path=./FontFiles/Poppins/,
    Extension = .ttf,
    Ligatures = TeX,
    UprightFont=*-Regular,
    BoldFont=*-SemiBold,
    ItalicFont=*-Italic,
    BoldItalicFont=*-SemiBoldItalic
    ]

\newfontfamily{\sectionfont}{PublicSans}[
    Path=./FontFiles/PublicSans/,
    Extension = .otf,
    Ligatures = TeX,
    UprightFont=*-Regular,
    BoldFont=*-Bold,
    ItalicFont=*-Italic,
    BoldItalicFont=*-BoldItalic
    ]

% Escala del Font size
\usepackage{scalefnt}

% Estilo de texto para las url
\usepackage{url}
\renewcommand{\UrlFont}{\rmfamily}

% Set space
\usepackage{setspace}

% Small caps
\usepackage{graphicx}

\makeatletter
\newlength\fake@f
\newlength\fake@c
\def\fakesc#1{%
  \begingroup%
  \xdef\fake@name{\csname\curr@fontshape/\f@size\endcsname}%
  \fontsize{\fontdimen8\fake@name}{\baselineskip}\selectfont%
  \uppercase{#1}%
  \endgroup%
}
\makeatother
\newcommand\fauxsc[1]{\fauxschelper#1 \relax\relax}
\def\fauxschelper#1 #2\relax{%
  \fauxschelphelp#1\relax\relax%
  \if\relax#2\relax\else\ \fauxschelper#2\relax\fi%
}
\def\Hscale{.94}\def\Vscale{.85}\def\Cscale{1}
\def\fauxschelphelp#1#2\relax{%
  \ifnum`#1=\lccode`#1\relax\scalebox{\Hscale}[\Vscale]{\char\uccode`#1}\else%
    \scalebox{\Cscale}[1]{#1}\fi%
  \ifx\relax#2\relax\else\fauxschelphelp#2\relax\fi}

% ##############
% Tikz
% ##############
\usepackage{tikz}

% ##############
% Imagenes y Figuras
% ##############
\usepackage{graphicx} % para insertar imagenes en Latex
\graphicspath{Images/} % le dice a latex donde están guardadas las figuras

% ##############
% SVG
% ##############
\usepackage[inkscapeformat=eps]{svg}

% ##############
% Tabla
% ##############
\usepackage{array}
\newcolumntype{L}[1]{>{\raggedright\arraybackslash}m{\dimexpr#1\linewidth-2\tabcolsep}}
\newcolumntype{R}[1]{>{\raggedleft\arraybackslash}m{\dimexpr#1\linewidth-2\tabcolsep}}
\newcolumntype{J}[1]{>{\justifying\arraybackslash}m{\dimexpr#1\linewidth-2\tabcolsep}}
\newcolumntype{C}[1]{>{\centering\arraybackslash}m{\dimexpr#1\linewidth-2\tabcolsep}}

% ##############
% Adjust Box
% ##############
\usepackage{adjustbox}

% ##############
% Icons
% ##############
\newcommand{\cvicon}[3]{\adjustbox{stack=cc}{\tikz [remember picture] {\node [opacity=#3]{\includesvg[width=#2]{SVG/#1.svg}}}}}

% ##############
% Itemize changing layout
% ##############
\usepackage{enumitem}
\setlist[itemize]{
    topsep = 0pt,
    before = {\partopsep = 0pt},
    itemsep = \parskip,
    parsep = 0pt,
    label = \mbox{-}\hfill,
    labelsep = 1ex,
    leftmargin = 1.75ex,
}


% ##############
% Hipervinculos
% ##############
\usepackage[hidelinks]{hyperref}
% \hypersetup{
%     pdftitle={DiegoGonzalez_CV},
%     pdfauthor={Diego Gonzalez}
% }
\hypersetup{pdfnewwindow}

% Url personalizado
\newcommand{\myurl}[1]{\adjustbox{cframe=linkcolor {2pt} {2pt}}{\url{#1}}}
\newcommand{\myhref}[2]{\adjustbox{cframe=linkcolor {2pt} {2pt}}{\href{#1}{#2}}}
\newcommand{\extlink}[1]{\adjustbox{stack=cc, cframe=linkcolor {1.5pt} {1pt}}{\href{#1}{\includesvg[width=2.25ex]{SVG/external-link.svg}}}}

% ##############
% CV Section
% ##############
\newcommand{\cvsection}[1]{\\[2\parskip]
\begin{tikzpicture} [remember picture]
    \node [
            rectangle, 
            rounded corners = 12pt,
            text width =\textwidth-8pt,
            fill = fillcolor,
            text = white,
            align=center,
            inner xsep = 4pt,
            inner ysep = 10pt,
            ] {\sectionfont\large\textbf{ \fauxsc{#1}}};
\end{tikzpicture}}

% ##############
% CV Item
% ##############
\usepackage{paracol}
\usepackage[most]{tcolorbox}

\newlength{\columnone}
    \setlength{\columnone}{.22\textwidth}
\newlength{\columntwo}
    \setlength{\columntwo}{\textwidth-\columnone}
    
\newcommand{\cvitem}[3]{
    \begin{minipage}[t]{\textwidth}
        \begin{minipage}[t]{\columnone}
            #1
        \end{minipage}
        \begin{minipage}[t]{\columntwo}
            \bfseries#2
        \end{minipage} 
    \end{minipage} \\
    \begin{minipage}[t]{\textwidth}
        \hspace{\columnone}
        \begin{minipage}[t]{\columntwo}
            \raggedright
            #3
        \end{minipage}
    \end{minipage}}

\newcommand{\cvitemtwo}[2]{
    \columnratio{.18}
    \setlength{\columnsep}{27pt}
    \setlength{\parskip}{.5\baselineskip}
    % \setcolumnwidth{\columnone,\columntwo}
    \begin{paracol}{2}
        \begin{tcolorbox}[
            breakable,
            left = 0pt,
            right = 0pt,
            top = 0pt,
            bottom = 0pt,
            boxsep = 0pt,
            frame hidden,
            bicolor,
            boxrule = 0pt,
            colback = white,
        ]
            \textbf{#1}
        \end{tcolorbox}
        \switchcolumn
        \begin{tcolorbox}[
            breakable,
            left = 0pt,
            right = 0pt,
            top = 0pt,
            bottom = 0pt,
            boxsep = 0pt,
            frame hidden,
            bicolor,
            boxrule = 0pt,
            colback = white,
            after = {\par}
        ]
            \raggedright
            #2
        \end{tcolorbox}
    \end{paracol}}

% ##############
% Paquetes de Visualización para crear el documento
% ##############
% \usepackage{layout}
% \usepackage{showframe}
% \usepackage{lipsum}